\section{State of the art and knowledge gap filled by this proposal}

One of the most striking gaps in the knowledge of our universe is the nature of dark matter. All we know about dark matter so far comes from its gravitational and astrophysics observations and simulations; astrophysics also provides some tantalizing hints towards the existence of a new particle~\cite{HooperLeane} that are imperative to pursue alongside alternative theories~\footnote{Caveat about gravitation}. 

In the past decade, the presence of a new massive subatomic particle that interacts weakly with ordinary particles (weakly interacting massive particles, or WIMPs) has been the dominant paradigm to explain the particle nature of dark matter. These particles are postulated by many theories (e.g. supersymmetry) and could be observed at a variety of experiments and at the LHC. Part of the appeal of such theories stems from their ability to explain the entirety of the relic density of dark matter in a simple way~\cite{WIMPMiracle}. The search for WIMPs is far from over: their interactions with ordinary matter could be more rare and not yet accessible by current searches, and both collider and astrophysical searches plan to continue probing these models with their upgrades into the next decade and beyond~\cite{Astro2020, EuropeanStrategy}. 

Partly motivated by the constraints set on such particles by first-generation direct searches and by results obtained in the first phase of the LHC data taking, the DM community has recently started to generalize the flagship searches for these weakly interacting massive particles (WIMPs) by expanding the search program for particles with either stronger or much weaker interactions with SM particles than what predicted by WIMP theories~\cite{FIMPs, StronglyInteracting}, or much lighter particles~\cite{DarkPhotons, ALPs}, or much more massive objects (e.g. Primordial Black Holes)~\cite{GW paper}~\footnote{Such scenarios can also fit the measurements of the relic density of dark matter through different mechanisms~\cite{FreezeIn}}. 

A new community of experiments looking for non-WIMP scenarios has flourished~\cite{PBC} and recent breakthroughs in multimessenger astronomy are contributing to defining the landscape of DM searches~\cite{GWBertone}. 
It is the perfect opportunity for LHC experiments to complement those searches, and to connect results for a broad picture of more complex DM scenarios that will be necessary for a coherent picture of a breakthrough discovery. 

These are strong motivations for this project to further searches for WIMPs to unprecedented precision, and enable detection of different DM candidates with broad theoretical connections, such as new particles from dark/hidden sectors~\cite{HiddenSector}. 