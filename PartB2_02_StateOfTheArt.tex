\subsection{Dark matter and role of this project}
\smallskip

Many gravitational and cosmological observations~\cite{Bertone} point to the existence of dark matter, for which the SM does not yield any predictions. 

These lead to an estimate of the present amount of dark matter in our universe (relic density), even though there are no clues on the mechanism leading to how this amount was reached. The two most popular mechanisms are the so-called freeze-in and freeze-out mechanisms.
%where the amount of DM in the early universe increases or decreases to the current value respectively via direct or indirect connections of DM to SM particles. 
%maybe I don’t need freeze-in but better to mention I know it exists? 
If DM is a new particle, implementations of those mechanisms require either a direct or an indirect connection of DM to SM particles, allowing its observation in experiments made of ordinary matter. 

The WIMP has been one of the most popular search targets for DM: its interaction strenght comparable to the weak force and its TeV-scale mass range make WIMPs detectable by a variety of experiments while satisfying the relic density. 

Indirect detection experiments in space (ID) observe excesses of SM particles over astrophysical backgrounds due to WIMP annihilation in DM-rich regions. 
Direct detection experiments (DD) detect the interaction between incoming WIMPs and recoiling target nuclei within the detector. 
The LHC is a major player in the quest for DM: while unable to ascertain the cosmological provenance of DM-like particles~\footnote{Particles that look like DM in LHC experiments may decay after leaving the detector, with a lifetime incompatible with DM’s cosmological timescales.}, it is able to produce DM in controlled conditions for a deeper understanding of their interactions with the SM. 
DD, ID and LHC experiments, combined with  astrophysical observations~\cite{Annika} are complementary and all necessary for the confirmation of a DM discovery within the WIMP paradigm. 

This well-established complementarity, acknowledged since the early 2000s~\cite{Snowmass} and fully rooted in the LHC DM search program established during the course of my StG~\cite{DMWG},  only paints a partial picture of the thriving landscape that is the DM community. 
Depending on the SM-DM interaction strength as well as on the DM mass, a wealth of new experiments contribute to the quest for WIMP and non-WIMP DM. These range from accelerator-based to table-top experiments~\cite{PBC}, to experiments sensitive to gravitational wave signals~\cite{BertoneGW,Cirelli}.  
It is clear that, given the breadth of possible explanations for DM, an equally broad approach that involves all possible detection techniques is needed.

In REALDARK, I will pursue a series of sensitive searches for DM at the LHC targeting WIMP and non-WIMP DM theories. These results will be a further step in my career as an established leader in the DM community, enhancing my contributions to the exploitation of synergies and complementarities among the whole experimental and theoretical DM community. 

\subsection{Theoretical framework}
\smallskip

The complementarity among different experiments in terms of common discovery potential requires a common theoretical ground. 

Theoretical models including DM particle candidates range from complex yet fully-specified theories (e.g. supersymmetry, or SUSY) to effective field theories at energy scales where the exact details interaction can be ignored. 

During the course of my StG, I have focused the LHC DM community around \textbf{simplified models}~\cite{DMWG}.
The main feature of these DM models is the introduction of a new particle (Beyond the Standard Model, BSM) that acts as the mediator of the interaction between ordinary matter particles and particles in the dark sector, as a step forward beyond EFTs. These models can then be used as building blocks for more complete theories. 

In their simplicity, these models are still well-motivated benchmarks for LHC searches. They can reproduce relevant experimental features with a limited number of parameters, allowing systematic exploration of the parameter space at a time where no experimental hints are observed. 
Furthermore, motivating searches for only a handful of new particles with simple interactions even if they are part of a complex spectrum is conceptually supported by how ordinary particles were uncovered in the history of the SM: the first particles discovered were the most common ones, and further complexity was unraveled at a later date. 

One of the most popular DM benchmark models of this kind contains a BSM vector particle analogous to the Z boson (a $Z’$ boson) to mediate the interaction between SM and DM particles, as shown in Fig.~\cite{fig:SimplifiedModels} (left). Additional interactions and modifications are required to make the model self-consistent~\cite{AnomalyFreeModels}, but they do not change its main experimental feature: given that the $Z’$ boson has been produced from the interaction of SM particles (e.g. quarks), then it can also decay into SM particles. 
This gives LHC experiments a unique insight into the dark sector: astrophysical hints of a $Z’$-mediated DM particle could be also characterized in terms of the interactions between the $Z’$ and the SM particles it decays into. Furthermore, the LHC would be sensitive to these decays even if the DM particle was inaccessible, and point the way to other means of verifying a possible $Z’$’s connection to DM. 

Visible decays of the  $Z’$ also connect DM mediator searches with the well-known class of searches for new resonances at particle colliders. New resonant particles have a large array of theoretical motivations beyond DM~\cite{Craig2bodyResonances} and are not yet fully constrained by existing searches. 

In particular, hadronically-decaying resonances with masses around the electroweak scale ($\approx$~100-200 GeV) have been very difficult to probe  due to the large amounts of backgrounds overwhelming the experiment’s data storage resources. This is a region where the weak force mediators reside, and that is also favored in WIMP and non-WIMP $Z’$ models that provide an explanation to indirect detection in terms of DM annihilation~\cite{HooperLeane}.  
To date, current searches are only sensitive to coupling strengths well above those of the weak force. This harms the joint discovery potential of upcoming direct detection experiments and LHC experiments~\cite{AnomalyFree}. The next sections describe how LHC experiments have begun to address this issue, and how preliminary results lead the way to the searches planned in WP2 and WP3 within this proposal. Another theoretically motivated possibility is that the mediator is a scalar, much like a new Higgs boson. While this proposal does not focus on this possibility, it paves the way for more sensitive searches for scalar mediators, which will be performed by the collaborators listed in Part B1.  

In parallel to considering simple $Z’$-mediated DM models, it is also imperative to consider alternative models that can have escaped detection so far. An example is a class of models where the mediator of the interaction does not directly connect DM and SM, but rather provides a \textbf{hidden portal} particle where the interactions between the are much weaker than in WIMP models.  One such portal particle is the so-called \textit{dark photon}~\footnote{This nomenclature may sound misleading as the dark photon is massive, unlike the SM photon, but we adopt it nevertheless to avoid confusion with the $Z’$ model which can also be considered a \textit{dark boson}}. The dark photon mixes with the ordinary photon and has to extremely feeble couplings to electrically charged particles as a consequence of that, rather than interacting directly with them particles~\cite{Holden,Curtin}. Models including dark photons theorize DM candidates much lighter than WIMPs, and the dark photons themselves are lighter than $Z’$. 

Together or individually, $Z’$ and dark photons can also be part of a “hidden” dark sector set of particles that cannot directly interact with the SM and also contains DM particles. This is one of the cases in which simplified models are used as building blocks for more complex models.

The theoretical and experimental ground for hidden sector models is much less explored than WIMPs, due to the additional complexity in the parameter space and to their non-standard detector signatures requiring new, dedicated experimental techniques. 
For this reason, systematic exploration of these models does not proceed by scanning the theory’s parameter space, but rather by mapping the characteristics of different possible configurations of parameters and ensuring that no experimental stone is left unturned. 
Searches for this kind of models have recently started taking shape at the LHC, in parallel to a vigorous pursuit by planned and existing non-LHC experiments~\cite{PBC}. 

In WP4, I will make use of my expertise in the reconstruction of hadronic jets and non-standard data taking technique to search for two classes of theoretically-motivated DM models that include a new strong force within the dark sector (dark QCD), using novel data taking techniques necessary for uncovering their signatures. 
With these searches, this proposal will contribute to the systematic mapping of the territory of hidden sector signature, in synergy with other LHC and non-LHC searches. 



%\section{State of the art and knowledge gap filled by this proposal}

%One of the most striking gaps in the knowledge of our universe is the nature of dark matter. All we know about dark matter so far comes from its gravitational and astrophysics observations and simulations; astrophysics also provides some tantalizing hints towards the existence of a new particle~\cite{HooperLeane} that are imperative to pursue alongside alternative theories~\footnote{Caveat about gravitation}. 

%In the past decade, the presence of a new massive subatomic particle that interacts weakly with ordinary particles (weakly interacting massive particles, or WIMPs) has been the dominant paradigm to explain the particle nature of dark matter. These particles are postulated by many theories (e.g. supersymmetry) and could be observed at a variety of experiments and at the LHC. Part of the appeal of such theories stems from their ability to explain the entirety of the relic density of dark matter in a simple way~\cite{WIMPMiracle}. The search for WIMPs is far from over: their interactions with ordinary matter could be more rare and not yet accessible by current searches, and both collider and astrophysical searches plan to continue probing these models with their upgrades into the next decade and beyond~\cite{Astro2020, EuropeanStrategy}. 

%Partly motivated by the constraints set on such particles by first-generation direct searches and by results obtained in the first phase of the LHC data taking, the DM community has recently started to generalize the flagship searches for these weakly interacting massive particles (WIMPs) by expanding the search program for particles with either stronger or much weaker interactions with SM particles than what predicted by WIMP theories~\cite{FIMPs, StronglyInteracting}, or much lighter particles~\cite{DarkPhotons, ALPs}, or much more massive objects (e.g. Primordial Black Holes)~\cite{GW paper}~\footnote{Such scenarios can also fit the measurements of the relic density of dark matter through different mechanisms~\cite{FreezeIn}}. 

%A new community of experiments looking for non-WIMP scenarios has flourished~\cite{PBC} and recent breakthroughs in multimessenger astronomy are contributing to defining the landscape of DM searches~\cite{GWBertone}. 
%It is the perfect opportunity for LHC experiments to complement those searches, and to connect results for a broad picture of more complex DM scenarios that will be necessary for a coherent picture of a breakthrough discovery. 

%These are strong motivations for this project to further searches for WIMPs to unprecedented precision, and enable detection of different DM candidates with broad theoretical connections, such as new particles from dark/hidden sectors~\cite{HiddenSector}. 