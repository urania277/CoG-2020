%Together or individually, mediator particles above and dark photons can also be part of a dark sector set of particles that do not interact significantly with the SM, including the DM candidate~\cite{Felix}. 

%Another possible story (but longer)
%Building blocks: dark QED
%Why attractive
%Conceptual connection to vector mediator above
%Why neglected
%Next-on: dark QCD
%Why attractive
%Why neglected




%For this reason, systematic exploration of these models does not proceed by scanning the theory’s parameter space, but rather by mapping the characteristics of different possible configurations of parameters and ensuring that no experimental stone is left unturned. 
%Searches for this kind of models have recently started taking shape at the LHC, in parallel to a vigorous pursuit by planned and existing non-LHC experiments~\cite{PBC}. 



%DM can't catch light DM even though there are a lot of tries
%So we go after the mediators again

%- Large experimental variability
%- Many motivations for lepton-jet models, but they neglect hadronic dark sector particles (eg SUSY lepton-jets below)
%- So we start where we can, and where we have motivations - B-L? 

%Peter's suggestion for B1, but I don't know where to add it? 
%The only thing I miss and I can understand that it can be hard to fit in is a motivation of WIMPs and
%in particular dark QCD. Could one maybe add just a little a la:
%The nature of dark matter is difficult to understand, because even today there is five times more
%dark matter than normal matter, this is only after essentially all normal matter in the Universe
%annihilated microseconds after the Big Bang (only 1 in 10,000,000,000 protons survived). The WIMP
%solution is a weakly coupled particle that yields the observed DM abundances by being weakly
%produced. It is attractive to search for WIMPs at the LHC because the energy of the accelerator
%allows studies above and below the electroweak phase transition. In Dark QCD one instead assumes that
%the abundances are similar because there also was an annihilation in the dark sector of the Universe
%and so Dark QCD models can potentially also explain the matter-antimatter symmetry observed in the
%Universe. It is attractive to search for Dark QCD at the LHC because it is a QCD collider and so one
%could expect that if there is a coupling between QCD and Dark QCD then we are likely to produce Dark
%QCD mesons and baryons at the LHC.%Check the latter

%SUSY lepton-jets: https://arxiv.org/pdf/0909.0290.pdf
%The cascades to N ̃1 may also result in colored particles and hence QCD-jets, but for the purpose of the current study we will assume that a substantial fraction of such cascades result in no colored particles. This is not a strong assumption as it is satisfied in many concrete examples of MSSM spectra [11], and may even be discarded altogether in actual lepton jet searches including QCD-jets.

%Existing text
%, and as a consequence it is only extremely feebly connected to electrically charged particles - I should probably talk about its decays
%too ambiguous? it decays into fermions as if it were a photon
%rather than direct interactions with


%This is one of the cases in which simplified models are used as building blocks for more complex models.


%In WP4, I will make use of my expertise in the reconstruction of hadronic jets and non-standard data taking technique to search for two classes of theoretically-motivated DM models that include a new strong force within the dark sector (dark QCD), using novel data taking techniques necessary for uncovering their signatures. 
%With these searches, this proposal will contribute to the systematic mapping of the territory of hidden sector signature, in synergy with other LHC and non-LHC searches. 