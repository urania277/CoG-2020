%WORRD SALLADDD
\subsubsection*{WP3: WIMP dark matter mediator searches}

\begin{wrapfigure}{L}{0.25\textwidth} \includegraphics[width=0.25\textwidth]{figs/WIMPMediator}
\caption{\label{fig:WIMPMediator} \small Sketch of WIMP mediator model \scriptsize \color{red} Improve quality, consider adding DD/ID etc. \color{black}}
\end{wrapfigure}

This model is used by the majority of DM LHC searches and can reproduce the observed relic density. It can be used as a building block for more complex theories (e.g. those mentioned in WP4) since it can describe the behavior of these processes even with a small number of parameters. 


Both classes of models can explain the amount of dark matter in the universe from cosmological observations (the DM \textit{relic density}). 

%In this proposal I will map these discoveries to two classes of models which explain the particle nature of dark matter discussed in WP3 and WP4. %too kitchen sink
The data collected using these techniques represents an entirely new dataset that can extend the experimental reach of the ATLAS experiment with a similar amount of resources~\cite{Resonances}. 



The contextualization and dissemination of the results obtained will not be limited to high energy physics, as it will be extended further afield, to other experiments searching for DM. 

%and thanks to the presence of a strong theory group with world-leading expertise in the DM models I intend to look for. 

\subsubsection*{WP2: Early searches and measurements}



\subsubsection*{WP3: Dark Matter mediator searches}

%One of the most striking gaps in the knowledge of our universe is the nature of dark matter. All we know about dark matter so far comes from its gravitational and astrophysics observations and simulations; astrophysics also provides some tantalizing hints towards the existence of a new particle~\cite{HooperLeane} that are imperative to pursue~\footnote{Caveat about gravitation}. 

This WP furthers searches for WIMP models at the LHC using the Trigger Level Analysis technique, in a scenario that is not fully constrained by existing searches, by searching for the visible decays of the particles mediating the interaction between ordinary and dark matter. 

\begin{wrapfigure}{L}{0.25\textwidth} \includegraphics[width=0.25\textwidth]{figs/WIMPMediator}
\caption{\label{fig:WIMPMediator} \small Sketch of WIMP mediator model \scriptsize \color{red} Improve quality, consider adding DD/ID etc. \color{black}}
\end{wrapfigure}

%WIMP DM can be produced via interactions of SM particles at the LHC through a new force mediator (analogous to the W and Z for the Weak Force), as in Fig.~\ref{fig:WIMPMediator}. Here, quarks within the colliding protons couple to a mediator particle which decays to a pair of WIMPs or, through the same interaction responsible for mediator production, to two jets. 
This model is used by the majority of DM LHC searches and can reproduce the observed relic density. 

In WP3, I will search for the decays of mediators of DM in two-jet final states, accompanied by a photon or a jet. I have pioneered LHC searches for this signature in my StG using traditional data-taking techniques as a way to look for new particles with lower masses with respect to what was possible in two-jet searches with the Trigger Level Analysis, and in this CoG I will implement the TLA technique in this kind of search, relying on the first ATLAS availability of TLA photons (\textbf{WP1}). This will lead to a major improvement to the sensitivity of particles in a region of phase space where the SM massive force mediators reside ($<$ 400 GeV), that is inaccessible with the same sensitivity by other kinds of searches, and that is motivated by a number of theoretical models beyond WIMP DM~\cite{ALP, HooperLeane}.  

\subsubsection*{WP4: Mapping dark/hidden sectors}

%Partly motivated by the constraints set on such particles by first-generation direct searches, and by the analyses I have led in the first phase of the LHC data taking, the DM community has recently started to generalize the flagship searches for these weakly interacting massive particles (WIMPs) by expanding the search program for particles with much weaker interactions with SM particles than what predicted by WIMP theories~\cite{FIMPs}. 
%This is a strong motivation for this project to enable detection of different DM candidates with broad theoretical connections, such as new particles from dark/hidden sectors~\cite{HiddenSector}. 

%While the searches in WP2 are powerful probes of WIMPs, they are not optimally sensitive to DM mediators whose interaction with the SM is even feebler. This is the case for “dark sector” or “hidden valley” models\cite{Zurek}, where the only connection between the SM and the DM particle spectrum occurs via a feebly coupled mediator particle (e.g. a photon-like particle with non-zero mass). Hidden valley models also postulate a multitude of dark sector particles, mirroring the complexity of the SM in a theory similar to the strong force.  
%These "dark Quantum Chromodynamics” (dark QCD) theories, where the fundamental constituents (e.g. dark quarks) fragment into a mixture of visible and invisible particles (dark hadrons), would still give rise to jets of particles. However, the particles forming the jet may be invisible, or only emerge after having partially traversed the detector material. Dark sector jets may also contain very light dark matter mediators that decay into low-energy electrons and muons. 
%Given the complexity of the parameter space driving the phenomenology of dark QCD, there are a wide range of possibility of how dark jets appear in the detector that need to be mapped. The majority of those possibilities currently escapes detection and are discarded at the trigger level, either because of the very large SM QCD  backgrounds (a problem shared by the search targets in WP3) or because the trigger system is unable to recognize their characteristic features in time and considers them noise.  
%The combination of TLA and partial event reconstruction is the solution to this problem: by recording a limited set of information reconstructed at the trigger level in addition to the full set of raw data behind the dark jet, we bypass storage limitations and we can reconstruct the features that distinguish signal from background at a later stage where more resources are available. 

While the dataset collected with those techniques allows mapping a large number of dark sector models, two classes that have not been searched at the LHC before are chosen to be investigated with Run-3 data in this project.

In the first class, one of the dark sector particles within the dark jet can easily make up the relic density of dark matter and is therefore completely invisible to the detector, leading to a \textit{semi-visible} jet. Since the main background for this signal is composed of mismeasured QCD jets, it is crucial that both QCD jets and dark jets are calibrated so that the measurement error and jitter can be minimized. As an expert in hadronic jets and their calibration, I will develop and employ techniques tailored for this specific search. 
In the second class of models, low-energy leptons from the decays of e.g. a light dark mediator or from a dark Higgs boson are an integral part of the dark jet, mirroring the way as QCD particle showers develop an interleaved electromagnetic component. The main challenge of searches for these models is the development of reconstruction algorithms and calibrations for these leptons, and we will rely on existing algorithm that the ATLAS collaboration has already developed but not yet widely used for physics measurements and implement them in the trigger level reconstruction within WP1. 

As part of WP5, the parameter space that will be investigated in these models will be carefully chosen by considering the constraints from past searches for the individual particles within the dark jet (e.g.~\cite{DarkPion})  and LHC measurements of the development of QCD jets that are sensitive to changes in the structure of jets, using a software program that enables the use of precision results released by the ATLAS, CMS and LHCb collaboration to be used to constrain the parameter space of new physics models~\cite{CONTUR}. 

\subsubsection*{WP5: Conceptualization, contextualization, synergies}

%\begin{wrapfigure}{R}{0.5\textwidth} 
%\begin{figure}[h!]
%\begin{center}
%\includegraphics[width=0.7\textwidth]{figs/flavorprobes.pdf}
%\caption{\label{fig:scales} Energy scales of generic NP models probed by heavy flavor observables. The energies accessible to searches at the LHC and a potential $100~\TeV$ future collider (FCC) are indicated. Figure adapted from~\cite{neubert:scales}}
%\end{center}
%\end{wrapfigure}
%\end{figure}


\section{Timeliness of the research program and appropriateness of the PI and team} 
\smallskip

This research program spans the entire upcoming LHC data-taking period (Run-3). The 2021-2026 period is the ideal time to advance the state-of-the-art in processing of large datasets and DM searches, ensuring continued impact through a significant extension of my current successful StG research program that relies on novel data-taking techniques. The LHC schedule includes an initial commissioning period where innovations can be deployed and tested via early measurements with start-up data, which forms the first part of this project. The second phase of this proposal will use the wealth of data delivered by the LHC during the fully-commissioned data-taking period for novel DM searches, with a dataset that will be more than twice as sensitive to the physics observables of this proposal than the data collected so far. 

I am uniquely suited to deliver an ambitious and timely research program in which I will lead a research team of two postdoctoral researchers and two students, complemented by talented Lund University undergraduates that I have a track record of recruiting and training.
As evidenced by my CV and track record, my profile combines both technical and scientific proficiency with leadership of large groups of scientists in both physics and data analysis tools. 
With my international collaborators and within my StG, I have led a paradigm shift in data-taking techniques in ATLAS from software concept to publication \textbf{[and I am now HSF convenor]}. I have authored a number of LHC searches for DM and new phenomena, and I have coordinated ATLAS- and LHC-wide working groups instrumental to the design of DM search strategies~\cite{DMWG}. 
As an internationally recognised expert in the field, I have been invited to write review articles on the subject area of this proposal~\cite{Buchmueller:2017qhf, Boveia:2018yeb}, and have contributed to the European Strategy update, which defines the next 10 years of Europe-wide and international research in HEP, as one of the scientific secretaries of both the Dark Matter and Beyond the Standard Model Physics Planning Groups~\cite{Strategy:2019vxc}.


% Mention who asked you to do this rather than saying who you're doing this with
%In addition to further enhancing my standing as an expert in DM, 

As a recognized expert in DM searches and collider data-taking techniques with a track record of synergistic interdisciplinary activities, the impact of this proposal will not be limited to high energy physics but will be disseminated further afield to generate impact in DM direct detection and cosmological disciplines.  

\section{Project planning}
\smallskip

Minigantt

I will lead a research team consisting of two postdoctoral researchers and two PhD students. 

\footnote{The first two years of Run-3 are the ideal opportunity to use the  to . 
The initial LHC accelerator plan for Run-3 foresees that the first two years (2021-2022) will be used as a ramp-up for the production data-taking period, with the majority of the dataset being collected in (2023-2024)~\cite{SomeRandomEuropeanStrategyPresentationOrTooMuch}.}.

%\begin{figure}[h!]
%\begin{center}
%\includegraphics[width=0.7\textwidth]{figs/timeline_new.pdf}
%\caption{\label{fig:timeline} Timeline of the proposal.}
%\end{center}
%\end{figure}

%The LHC will begin Run III with a commissioning period during %which the software trigger will be exercised. A detector %commissioning paper will be published in Q2 2021, and a %performance paper in Q4 2021 where the new exclusive trigger %selections will be used to highlight the increased efficiency %with respect to Run II. 


\section{Conclusions}

%CAN BE USED This proposal will solve this issue by enabling the ATLAS experiment to record data in a new manner. This will permit, for the first time, this wealth of recorded data to be used for early measurements and new dark matter searches, and propagates the results and the developed tools to the broader community. \textbf{Mention the tools somewhere}

\clearpage
%\addcontentsline{toc}{section}{References}
\setboolean{inbibliography}{true}
%\bibliographystyle{LHCb}
\bibliographystyle{JHEP}
\bibliography{researchrefs}


\clearpage

%
\section*{Section B: Curriculum Vitae} 
\subsection*{Personal information}
\begin{tabular}{rlrl}
{\bf Name} & Dr.~Conor Fitzpatrick & {\bf Date of birth} & 03/10/1982 \\
{\bf Nationality} & Irish \\
{\bf ORCiD} & \hyperlink{https://orcid.org/0000-0003-3674-0812}{0000-0003-3674-0812} & {\bf inSPIRE} & \hyperlink{https://inspirehep.net/author/profile/C.Fitzpatrick.1}{C.Fitzpatrick.1}

\end{tabular}
\subsection*{Education}
\begin{flushleft}
\begin{tabular}{rl}
\bf{2012} & PhD, Experimental Particle Physics \\ 
& School of Physics and Astronomy, University of Edinburgh, UK. \\ 
& Supervisor Prof. F. Muheim \\ \\
\bf{2008} & Master of Physics with honours \\ 
& School of Physics and Astronomy, University of Edinburgh, UK. \\ 
\end{tabular}
\end{flushleft}
\subsection*{Current position}
\begin{flushleft}
\begin{tabular}{rl}
\bf{2014 - \phantom{2014}} &  Collaborateur Scientifique, Laboratoire de physique des hautes \'energies\\ 
& \'Ecole polytechnique f\'ed\'erale de Lausanne, Lausanne, Switzerland
\end{tabular}
\end{flushleft}

\subsection*{Fellowships and awards}
\begin{flushleft}
\begin{tabular}{rp{13.5cm}}
\bf{2012 - 2014} & Research Fellowship, LHCb experiment \\ 
& CERN, Switzerland \\
\\
\bf{2016}\phantom{ - 2014} & LHCb Early Career Scientist Award  \\ 
& LHCb Collaboration, CERN \\ 
& ``...for having implemented and commissioned the revolutionary changes to the LHC Run-2  high-level trigger, including the first widespread deployment of real-time analysis  techniques in high-energy physics.''
\end{tabular}
\end{flushleft}
\subsection*{Supervision of graduate students}
\begin{flushleft}
\begin{tabular}{rp{13.5cm}}
\bf{2014 - 2018} &  \'Ecole polytechnique f\'ed\'erale de Lausanne, Lausanne, Switzerland \\
& {\bf{PhD student supervisor}} for Vincenzo Battista, EPFL Thesis no. 8848 `Measurement of time-dependent \CP violation in \HepProcess{\PB\to\PDmp\Ppipm} decays and optimisation of flavour tagging algorithms at LHCb'\\
& {\bf{Masters thesis supervisor}} for Marc Huwiler `A search for the decay \HepProcess{\PBzero\to\PDsplus\PDsminus} using multivariate techniques at LHCb'
\end{tabular}
\end{flushleft}
\subsection*{Teaching activities}
\begin{flushleft}
\begin{tabular}{rp{13.5cm}}
\bf{2014 - 2018} &  \'Ecole polytechnique f\'ed\'erale de Lausanne, Lausanne, Switzerland \\
& Course organiser `Introduction to High Energy Physics Software' for Masters and final year undergraduate students  \\ 
& Project Supervisor for final-year undergraduate and Masters student projects.\\
\bf{2013 - 2018} & CERN, Meyrin, Switzerland \\
& Summer student programme supervisor for four students, two of whom have {\textbf{received awards for their work}}. Supervisor for one Masters internship. \\
\bf{2008 - 2012} & University of Edinburgh, UK \\ 
& Tutor, laboratory and workshop demonstrator for introductory physics courses. \\%
  & Laboratory demonstrator for final year undergraduate course `Electronic Methods in the Physical Laboratory'. 
\end{tabular}
\end{flushleft}
\subsection*{Organisation of Scientific Meetings}
\begin{flushleft}
\begin{tabular}{rl}
\bf{2017} & QCD + Heavy Flavour session convenor, Rencontres de Blois  \\
\bf{2015} & Discussion leader, CERN-Fermilab Hadron Collider Physics summer school \\
\bf{2011} & Organising Committee, Young Experimentalists and Theorists Institute, IPPP Durham  \\
\end{tabular}
\end{flushleft}

\subsection*{Leadership Responsibilities} 
\begin{flushleft}
\begin{tabular}{rp{14cm}}
  \bf{2017 - \phantom{2018}} & \textsl{\textbf{Project Leader}, LHCb Trigger} \\
  & As Project Leader, I lead an international team of 9 collaborating institutes responsible for the operation of the present LHCb trigger, and R\&D for future trigger upgrades. \\
  \bf{2015 - 2017} & \textsl{\textbf{Convenor}, LHCb Beauty to Open Charm (B2OC) Working Group}  \\
    & The LHCb physics programme is organised into nine top-level physics working groups. As convenor of the largest working group I was responsible for the physics output of over 60 analysts working on \sim30 analyses. I led the analysers to the publication of 20 peer-reviewed papers including several world-first and most precise measurements. \\
  \bf{2016 - 2017} & \textsl{Deputy Project Leader, LHCb Higher Level Trigger} \\
  \bf{2013 - 2015} & \textsl{Convenor, LHCb B2OC (Time-Dependent) subgroup}\\
  \bf{2012 - 2014} & \textsl{Simulation co-ordinator, LHCb Higher Level Trigger} \\
\end{tabular}
\end{flushleft}

~

\subsection*{Research statement}

My research career (2006 - present) has focused on precision measurements of fundamental particles and matter/antimatter asymmetries with the LHCb experiment. While I have developed a broad range of research outputs in several areas of the LHCb research programme, I have specialised in decay-time-dependent analyses through which I have made world-leading measurements of \CP violation in \PBz and \PBs mesons. In measurements of this kind, the SM can be quantified, for example through measurements of the CKM angle $\gamma$, and NP can be searched for, the determination of $\phi_s$ in \HepProcess{\PBs\to\PJpsi\Pphi} and \HepProcess{\PBs\to\PDsplus\PDsminus} decays being a perfect example. Driven by a desire to improve sensitivities for measurements of $\gamma$ and $\phi_s$, I have played a leading role in both the operation of the present LHCb Run 2 trigger and in defining the trigger for the first LHCb upgrade. My work in these areas have been recognised by the collaboration with a working group convenorship, an early career scientist award and Project Leadership of the LHCb higher level trigger. 

Funding of this project will allow me to take the next step in my career, and to become an established independent scientist with a permanent academic position.
It is my intention to become an internationally recognised leader in precision measurements at collider experiments, and an expert in detector trigger and data acquisition techniques. I am enthusiastic about building and leading a team that will develop my ideas, taking the RTA paradigm to future experiments, and with them derive new insights into the nature of our universe. 
 

\section*{Section B: Curriculum Vitae}

\clearpage

\section*{Appendix: All ongoing and submitted grants and funding of the PI}

%\begin{table}[ht]
    \centering
\begin{tabular}{p{3.5cm}|p{3cm}|p{1.4cm}|p{1.5cm}|p{1cm}|p{2.3cm}}
    Project title & Funding Source& Amount (EUR) & Period & Role of PI & Relation to current ERC Proposal \\ \hline\hline
    Precision tests of the Standard Model with doubly-charmed beauty decays & Science \& Technologies Facilities Council (STFC) UK & 751,000 & 10/2019-10/2024 & PI & see text \\ \hline
   Connecting the universe: Bringing Real Time Analysis to Particle Physics and Astronomy  & United Kingdom Research and Innovation (UKRI) & 1,635,664 & 04/2019-04/2023 & PI & see text \\ \hline
    Precision tests of the Standard Model with doubly-charmed beauty decays & Royal Society UK & 951,364 & 10/2019-10/2024 & PI & see text\\ \hline
\end{tabular}
\caption{Submitted grants involving the PI.}
    \label{tab:grants}
\end{table}

\noindent
At present I am not involved in any ongoing grants.  
Table~\ref{tab:grants} lists grants to which I have applied and for which a decision has yet to be made.

Of the three proposals listed, those for the Royal Society University Research fellowship (URF) and the STFC Ernest Rutherford fellowship (ERF) cover similar topics to those addressed in this proposal, however they are considerably less ambitious. The ERF and URF cannot be accepted simultaneously if both are successful. If both this proposal and one of the ERF/URF proposals are successful it would be possible to hold both simultaneously, with the other grant effectively bringing additional funding into the project, enabling an expansion of the scope and support of an additional PDRA to work on RTA commissioning and trigger R\&D for future upgrades. 
The UKRI Future Leaders Fellowship (FLF) has only minor overlap with this proposal, but is complementary as it is designed to develop the RTA paradigm on future large-scale experimental infrastructure (The Square Kilometer Array). If successful this would enable software engineering expertise to enhance and expand the RTA developments in WP1.

\clearpage

\section*{Section C: Early achievements track record}

%\subsection*{Publications} 
I am a named author of more than 400 papers published by the LHCb collaboration, in addition to several few-author papers. The full LHCb collaboration author list includes over 500 individuals (even more in recent papers), and therefore cannot be given here.  
Following the standard procedure in high energy physics, all authors are listed alphabetically.
As my Ph.D.\ supervisor is also a member of the LHCb collaboration, he is a co-author of mine on those publications, but not on my other papers.
A complete list of all my papers, including as-yet unpublished preprints, can be found at 
\begin{center}
    \href{http://inspirehep.net/author/profile/C.Fitzpatrick.1}{http://inspirehep.net/author/profile/C.Fitzpatrick.1}
\end{center}
The most relevant papers for this proposal are:
 \begin{enumerate}
     \setlength\itemsep{1ex}

          \item {\bf{ Measurement of \CP violation in \HepProcess{\PBz\to\PDmp\Ppipm} decays }} \\%
                {R.~Aaij {\it et al.} [ LHCb Collaboration ].}      \\%
                JHEP 06 (2018) 084 \\
                       I initiated this analysis, and together with a PhD student under my supervision, was responsible for the \HepProcess{\PBz} invariant mass fit, the development of the decay time fitting framework modifications necessary to extract the \CP observables, treatment of the opposite-side tagging parameterisation and the decay time resolution determination. I showed that the fit was sensitive to the flavor tagging calibration and did not require that this was measured independently, resulting in reduced systematic uncertainties.

        \item {\bf{Measurement of the CKM angle $\gamma$ from a combination of LHCb results}} \\%
                                   {R.~Aaij {\it et al.} [ LHCb Collaboration ].}      \\%
                  JHEP 12 (2016) 087 \\
            As Convenor of the \LHCb working group that measures $\gamma$, I was responsible for the underlying analyses culminating in this result, which, for the first time, was the most precise single-experiment determination of $\gamma$ from \LHCb.
        
    \item {\bf{Measurement of the \CP-violating phase \phis in \BsToDsDs decays}} \\%
            {R.~Aaij {\it et al.} [ LHCb Collaboration ].} \\%
            Phys. Rev. Lett. 113, 211801 (2014)\\
            I was the lead proponent of this publication, performing the entirety of the analysis from the offline selection onwards.
            I was responsible for taking the analysis and publication through all stages of review, both internal within the LHCb collaboration and after submission to the journal.

    \item {\bf{Prompt charm production in pp collisions at $\sqrt{s}= 7~\TeV$}}         \\%
    {R.~Aaij {\it et al.} [ LHCb Collaboration ].}      \\%
    Nucl. Phys. B871 (2013) 1-20.\\
     I made the \HepProcess{\PDspm} cross-section measurement detailed in this publication in addition to the \PDpm cross-check. This work built upon the \PDspm/\PDpm cross-section ratio I performed with the first \LHCb data as an early measurement, presented at the ICHEP conference in 2010.

    \item {\bf{Measurement of the CP-violating phase $\phi_s$ in the decay \BsToJpsiPhi}} \\%
         {R.~Aaij {\it et al.} [ LHCb Collaboration ].} \\%
         Phys. Rev. Lett. 108, 101803 (2012). \\
         This measurement was the main component of my PhD thesis. I developed the fitting framework, verified the helicity and transversity formalisms, and implemented the Feldman-Cousins statistical procedure used to determine the published result.

  \end{enumerate}

\subsection*{International conference talks} 
I have given the following talks at major international conferences:
  \begin{enumerate}
    \setlength\itemsep{1ex}
          \item {\bf Flavour at LHCb: Recent results and prospects }\\
        Seventh Workshop on Theory, Phenomenology and Experiments in Flavour Physics, Capri, (8 - 10 June 2018)
   \item {\bf Too much of a good thing: How to trigger in a signal-rich environment} \\
        EP/IT Data Science seminar series, CERN, (13 Dec. 2017)
    \item {\bf CP Violation} (review)\\
            $29^{\textrm{th}}$ Rencontres de Blois, Blois, France (28 May - 3 June 2017)
    \item {\bf Combining $\gamma$ at LHCb}\\
            $9^{\textrm{th}}$ International Workshop on the CKM Unitarity Triangle, Mumbai, India (28 Nov - 2 Dec 2016)
    \item {\bf LHCb 2015 Highlights and Status}\\
            $178^{\textrm{th}}$ session of the CERN Council, CERN, Geneva (14-18 Dec 2015)
    \item {\bf Measurement of \CP observables using \BsToDsDs at LHCb}\\
            $8^{\textrm{th}}$ International Workshop on the CKM Unitarity Triangle, Vienna, Austria (8-12 Sept 2014)
    \item {\bf The upgrade of the LHCb trigger system}\\
            Workshop on Intelligent Trackers, Philadelphia, PA (14-16 May 2014)
    \item {\bf Hadronic b decays and the Unitarity triangle angle $\gamma$ at \LHCb}\\
      Rencontres de Moriond, QCD and High Energy Interactions, La Thuile, Italy (22-29 March, 2014)
  \end{enumerate}

I have also taken part in a large number of local and international workshops on flavour physics that are not listed above. 
I was invited to serve as the QCD + Heavy Flavour Session Convenor, Rencontres de Blois 2017, and as discussion leader at the CERN-Fermilab Hadron Collider Physics Summer School, 2015.

\subsubsection*{LHCb Upgrade Trigger R\&D}
The \LHCb Run I and Run II trigger consists of a level-0 hardware on-detector stage that must reduce the rate of \pp crossings to the $1~\MHz$ limit at which the present detector readout operates, at which point a second software Higher Level Trigger (HLT) stage reduces this rate further to satisfy offline data processing and storage requirements. From Run III onwards LHCb will operate at five times the Run I instantaneous luminosity with several subdetector upgrades. As a CERN fellow I performed studies of what this increased luminosity would mean for the trigger~\cite{LHCb-PUB-2014-027}. My studies showed that the most cost-effective solution would be to read out the full detector at the collider bunch crossing frequency of $30\MHz$ directly into a software trigger where the full event can be reconstructed. As a result of this work, \LHCb chose this triggerless readout with a fully software trigger as the upgrade design~\cite{CERN-LHCC-2014-016}. \textbf{For this work I was made deputy project leader of the HLT responsible for the upgrade in 2016.}

\subsubsection*{The LHCb Run II Trigger commissioning and operations}
The LHCb Run II HLT serves both as a flexible software trigger for the LHCb physics programme, and as a testbed for the upcoming upgrade which will take data in 2021.
However, in Run II the signals needed by the broad LHCb physics programme are still subject to the $1~\MHz$ readout limit.
As a chercheur scientifique at EPFL and deputy project leader of the trigger, I understood the importance of maximising the efficient use of this rate, and developed a method to optimise the level-0 trigger configuration using a genetic algorithm. This optimisation maximises the signal efficiency subject to the readout constraint for the entire LHCb physics programme, taking into account the physics priorities of the experiment and detector deadtime. \textbf{My work as part of the HLT team on commissioning the Run II trigger was recognised with an \LHCb early career scientist award}, and as of 2017 I have been made Project Leader for the HLT, where I direct both the operational activities of the Run II HLT, and the construction of the upgrade trigger.

\end{document}  
