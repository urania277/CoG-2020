As outlined in part B1, this proposal is composed of 5 interconnected work packages. 
The proposal’s overarching aim is to discover or constrain the particle nature of dark matter via its production at the LHC and the contextualization of these results in the global DM theoretical and experimental landscape. 
The basis for reaching this aim is the implementation of data-taking techniques that enable the ATLAS experiment to obtain a much bigger physics output from the upcoming LHC dataset without requiring a significant increase in resources. 
Physics results and software tools resulting this proposal will be shared with the broader DM community. 

The main objectives of the \textsc{Realdark} WPs are: 

\begin{description}

%In \textsc{Realdark} we will use novel data taking techniques (developed in Work Package 1) to \textbf{further the systematic exploration} of the hadronic decays of the DM mediator as in the left-hand side of Fig.~\ref{fig:feynman} (WP2-3). 

\item[(WP1)]  \textbf{To extend the capabilities of the ATLAS trigger system} with a comprehensive set of real-time analysis techniques. 
%The first cornerstone of WP1 are the TLA technique and the Partial Event Building (PEB) technique, which overcomes the traditional paradigm of first recording all detector data and then analyzing it by performing the data analysis directly at the trigger level, so that the majority of raw detector data can be dropped. 
In WP1, we will:
\begin{enumerate} 
\item implement photons, electrons and muons in TLA, starting from the jet prototype developed in my StG;
\item contribute to a suite of reconstruction and calibration techniques for HLT objects that can also be used offline;
\item implement a combination of PEB and TLA techniques targeting complex detector signatures;
\item study ML data compression techniques to be used for TLA data and beyond in future LHC runs. 
\end{enumerate}
%We will also seek further gains in storage using machine learning algorithms for data compression. %this sentence makes no sense
\item[(WP2)] \textbf{To commission the upgraded ATLAS trigger with early Run-3 data}, using searches that already have a solid methodological basis from Run-2. In WP2 we will:
\begin{enumerate} 
\item study the performance of physics objects reconstructed and calibrated at the HLT and offline; 
\item deploy and validate new calibrations and real-time analysis techniques with well-established generic searches for new phenomena in the dijet spectrum;
\item prepare end-to-end analysis code to be used for faster analysis turnaround and as part of the dissemination strategy.  
% such as generic high- and low-mass dijet (TLA). 
\end{enumerate} 
%We will deploy  for new phenomena that can already lead to groundbreaking results in the earlier stages of LHC data taking. The outcomes of this WP will be physics results with a the new LHC data, as well as tools and publications measuring the performance of trigger and offline objects.  

\item[(WP3-WP4)] To use the LHC dataset recorded with novel data taking techniques to perform searches sensitive to two broad classes of DM models:
 
\begin{itemize} 
\item in WP3, we will search for hadronic decays of WIMP DM mediators in the dijet and dijet+ISR final states using a TLA that including both photons and jets, to probe electroweak-scale mediator masses not fully explored by traditional searches.
\item in WP4, we will search for dark matter candidates and dark photons within the jets as predicted from dark QCD, using a combination of TLA and PEB to recover sensitivity to signals that escape traditional detector reconstruction techniques.
\end{itemize}

\item[(WP5)] \textbf{To disseminate and communicate physics results and tools} to make them possible to the broader DM and experimental community. In WP5 we will: 

\begin{enumerate} 
\item collaborate with theory experts to identify the most promising parameter space for searches in WP4. 
\item contribute to organizing and leading the new cross-community initiative for DM that I co-founded in 2019 (\href{https://indico.cern.ch/e/iDMEu/}{iDMEu}), including nuclear physics, astroparticle physics and particle physics.
\item disseminate technical outcomes of WP1 to other experiments at the LHC and beyond. 
\item disseminate the outcome of the searches in WP3 and WP4, whether they are a discovery to be characterized or constraints that will guide future DM experiments. The latter two objectives will be supported by my role in the \href{https://hepsoftwarefoundation.org}{HEP Software Foundation} and by my involvement in the \href{https://projectescape.eu}{ESCAPE project}, as detailed below. 
\end{enumerate} 

\end{description}

%%%

The theoretical and experimental ground for hidden sector models is much less explored than for WIMPs, 
due to additional complexity in the parameter space and to non-standard detector signatures requiring new, dedicated experimental techniques. 


