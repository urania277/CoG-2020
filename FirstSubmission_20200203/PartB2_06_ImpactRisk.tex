The data-taking techniques in this proposal have only been tested in ATLAS on preliminary use cases. 
While the experience that I built over the course of the past 5 years of my StG reassures on the feasibility of the outcomes of this proposal, there are certain risks associated to it. %Why/how/why should anyone bother
They are listed below.  
%\begin{table}
%\small
%\caption{\small Implementation risks\label{tab:RiskManagement}}
\begin{center}%\scriptsize
%\begin{center}
%\resizebox {\textwidth }{!}{%
\begin{tabular}{p{60mm}p{20}p{85mm}}
\toprule
%Objective: & \multicolumn{3}{|p{4cm}|}{\pbox{8cm}{\color{blue}{Main training events and conferences}}} & 
\midrule
\textbf{Risk} & \textbf{Risk factor} &  \textbf{Mitigation} \\
Delays in recruitment of team members with the right profile & Low & Seeking candidates within the trigger community, announcement of vacancies earlier than or soon after signature of grant agreement. \\
Delays in LHC schedule & Medium & Prototyping trigger improvements and searches with cosmic ray data, Run-2 data and simulation. Modifying work schedule so that end-to-end software implementation happens before trigger validation.\\
%HLT tracking not sufficiently performant to be ran for all events for WP3 and WP4 searches & Medium & Software engineer working full-time on tracking, purchase of HLT nodes instead of storage servers to increase HLT farm capacity (costing already done prior to submission). We will seek funding for the storage servers at a later date, and if that is not available we will use WLCG to store data in WP4 instead of local storage. This in turn may delay the publication of the last WP4 beyond the end of this project, requiring a no-cost extension. %this is dangerzone
%\\
\bottomrule
\end{tabular}
%}
\end{center}
%\vspace{-5mm}
%\end{table} 

%%%%%%BEGIN TO COMPILE UP TO HERE 
\clearpage
\begingroup
    \setboolean{inbibliography}{true}
\bibliographystyle{LHCb}
    \linespread{0.9}\selectfont
\bibliography{researchrefs}
\endgroup
%{\bf Note:} The PI is an author of Refs.~\cite{LHCb-PUB-2014-027,Bourgeois:2018nvk} and all the LHCb collaboration papers, with particular contributions to  Refs.~\cite{Aaij:2014jba,CERN-LHCC-2014-016,Aaij:2013mga,Aaij:2014ywt,LHCb-PUB-2014-040,Aaij:2016kjh,Aaij:2017lff}.
%The PI is acknowledged in Ref.~\cite{Jung:2014jfa}.
\end{document}  
%%%%%%END TO COMPILE UP TO HERE 
