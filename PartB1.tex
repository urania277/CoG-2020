%NOTE: This doc formatted according to http://ec.europa.eu/research/participants/data/ref/h2020/call_ptef/pt/2018-2020/h2020-call-pt-erc-stg-2019_en.pdf



\documentclass[11pt,a4paper]{article}
\usepackage[left=2.0cm,top=2.0cm,right=2.0cm,bottom=1.5cm]{geometry}               
%\usepackage[subtle]{savetrees}
\usepackage[bibbreaks=normal, paragraphs=normal, floats=tight, mathspacing=tight, lists=tight, title=tight, margins=normal, wordspacing=normal, tracking=normal, charwidths=normal, bibnotes=normal, mathdisplays=tight, leading=tight, indent=tight, bibliography=normal, sections=tight]{savetrees} 

\usepackage{graphicx}
\usepackage{amssymb}
\usepackage{epstopdf}
\usepackage{xspace}     
\usepackage{wrapfig}

\usepackage{color}
\usepackage{colortbl}
\usepackage{amsmath} % Adds a large collection of math symbols                  
\usepackage{ifthen} % for conditional statements               
\usepackage{amssymb}
\usepackage{amsfonts}
\usepackage{upgreek} % Adds in support for greek letters in roman typeset       
\usepackage{titling}
\usepackage{makecell}
\usepackage{pgfgantt}
\usepackage{lscape}
\usepackage{multirow}
\usepackage{titlesec}
\usepackage{url,amsmath,booktabs,hepunits,abhepexpt,abhep,xcolor}
\usepackage[colorlinks]{hyperref}    % Hyperlinks in references
\usepackage[all]{hypcap} % Internal hyperlinks to floats.
   

\newboolean{uprightparticles}
\setboolean{uprightparticles}{false} %Set to true to get roman particle symbols

%If we need more space we can investigate: https://tex.stackexchange.com/questions/273086/use-smaller-headheight-in-fancyhdr, https://tex.stackexchange.com/questions/271159/turn-off-fancyhdr-auto-spacing
\usepackage{fancyhdr}

\usepackage{rotating}

%\usepackage{fancyheadings}
\pagestyle{fancy}


% Playing with the page size 

\usepackage{array}
\newcolumntype{y}[1]{>{\raggedleft\arraybackslash}p{#1}}

%\renewcommand*{\arraystretch}{1.2}

%TB
%\addtolength{\oddsidemargin}{-10pt}
%\addtolength{\evensidemargin}{-10pt}
%\addtolength{\textwidth}{25pt}
%\addtolength{\textheight}{76pt}
%end TB

%\addtolength{\textfloatsep}{-2pt}
\setlength{\droptitle}{-44mm}

\setlength{\headwidth}{\textwidth}


\newboolean{articletitles}
\setboolean{articletitles}{false}

\usepackage{cite}
\usepackage{mciteplus}

\newboolean{inbibliography}

\titlespacing{\section}{0pt}{4pt}{0pt}
\titlespacing{\subsection}{0pt}{4pt}{0pt}
\titlespacing{\subsubsection}{0pt}{4pt}{0pt}

%\titlespacing{\section}{0pt}{\parskip}{0pt}
%\titlespacing{\subsection}{0pt}{\parskip}{0pt}
%\titlespacing{\subsubsection}{0pt}{\parskip}{0pt}


\title{{\Large Case for ERC Consolidator grant}}
\author{{\normalsize Caterina Doglioni}}
\date{}                                           % Activate to display a given date or no date

\usepackage[parfill]{parskip}

\begin{document}
\lhead{{\small Doglioni}}
\chead{{\small Part B1}}
\rhead{{\small REALDARK}}
\begin{center} 

{\Large\bf ERC Consolidator Grant 2020} \\
	{\Large\bf Research Proposal [Part B1]}  \\
 
\vspace{2cm} 
{\huge {\bf }}   \smallskip  

\vspace{2cm} 
{\Huge{REALDARK}} \\ 
\vspace{1cm} 
\vspace{1cm}
\end{center} 
\begin{tabular}{rcl}
Principal Investigator & : & Dr.~Caterina~Doglioni \\
Host Institution & : & Lund University \\ 
Proposal duration in months & : & 60 \\
\end{tabular}  
\vspace{2cm}


\begin{center} {\bf Summary}  \end{center}
%Up to 2000 chars...
Add summary here
\clearpage

\section*{Section A: Extended synopsis of the proposal} 

\medskip

\color{blue} The nature of 85\% of the matter in the universe is unknown. I will use data recorded with innovative techniques at the ATLAS experiment for a broad program of searches for this dark matter, providing valuable input to the community of dark matter searches worldwide, in terms of groundbreaking technological advancements in data taking and physics breakthroughs. \color{black}

The first years of data taking at the Large Hadron Collider (LHC)~\cite{LHC2008} at CERN, yielded the discovery of a new fundamental particle, the Higgs boson~\cite{Khachatryan:2016vau}. LHC data confirmed the predictive power of the Standard Model (SM) of particle physics, the theory of fundamental particles and non-gravitational interactions. However, the amount of ordinary matter described by the SM is exceeded by a kind of unknown matter found by cosmological observations, called Dark Matter (DM)~\cite{Bertone:2016nfn}. 
According to theoretical hypotheses accounting for the measured amount of DM in the universe with a new particle, DM could be created at the LHC in collisions of ordinary particles, with results complementing experiments directly looking for cosmological DM~\cite{Boveia:2018yeb}. 

However, no hints of this kind of DM have been found so far.
While it is worthwhile to continue these search DM since its interactions may be more rare and not yet accessible with the current LHC dataset, we need to significantly broaden our field of view ~\cite{Bertone:2018xtm} in order to elucidate the particle nature of DM and its interactions. In this Consolidator Grant, I propose to do so for LHC searches in the following ways:

\begin{itemize}
	\item by making the most of LHC data with innovative data-taking techniques, overcoming technological limitations that force experiments to discard most data milliseconds after being taken;
	\item  by searching for more complex, less-explored yet theoretically motivated DM models that may have escaped being recorded and detected in this first phase of LHC data taking;
	\item by connecting to the broader DM community that includes the wealth of astroparticle and non-collider experiments planned to start taking data in this decade, sharing physics results but also working together on tools for data processing, analysis and open science. 
\end{itemize}

%\begin{wrapfigure}{R}{0.5\textwidth} 
%\begin{figure}[h!]
%\begin{center}
%\includegraphics[width=0.7\textwidth]{figs/flavorprobes.pdf}
%\caption{\label{fig:scales} Energy scales of generic NP models probed by heavy flavor observables. The energies accessible to searches at the LHC and a potential $100~\TeV$ future collider (FCC) are indicated. Figure adapted from~\cite{neubert:scales}}
%\end{center}
%\end{wrapfigure}
%\end{figure}



\section{Project description}
\subsection{WP1: ...}

\section{Project planning}
I will lead a research team consisting of two postdoctoral researchers and two PhD students. 

%\begin{figure}[h!]
%\begin{center}
%\includegraphics[width=0.7\textwidth]{figs/timeline_new.pdf}
%\caption{\label{fig:timeline} Timeline of the proposal.}
%\end{center}
%\end{figure}

%The LHC will begin Run III with a commissioning period during %which the software trigger will be exercised. A detector %commissioning paper will be published in Q2 2021, and a %performance paper in Q4 2021 where the new exclusive trigger %selections will be used to highlight the increased efficiency %with respect to Run II. 



\clearpage
%\addcontentsline{toc}{section}{References}
\setboolean{inbibliography}{true}
\bibliographystyle{LHCb}
\bibliography{researchrefs}

~ 

%{\bf Note:} The PI is acknowledged in Ref.~\cite{Jung:2014jfa} and is an author of Refs.~\cite{LHCb-PUB-2014-027,Williams:1670992} as well as of all publications by the LHCb collaboration -- Refs.~\cite{Alves:2008zz,CERN-LHCC-2014-016,LHCb:2011aa,Aaij:2013mga,Aaij:2014ywt}.
%The PI is the contact author within LHCb for Ref.~\cite{Aaij:2014ywt}.

\clearpage

%
\section*{Section B: Curriculum Vitae} 
\subsection*{Personal information}
\begin{tabular}{rlrl}
{\bf Name} & Dr.~Conor Fitzpatrick & {\bf Date of birth} & 03/10/1982 \\
{\bf Nationality} & Irish \\
{\bf ORCiD} & \hyperlink{https://orcid.org/0000-0003-3674-0812}{0000-0003-3674-0812} & {\bf inSPIRE} & \hyperlink{https://inspirehep.net/author/profile/C.Fitzpatrick.1}{C.Fitzpatrick.1}

\end{tabular}
\subsection*{Education}
\begin{flushleft}
\begin{tabular}{rl}
\bf{2012} & PhD, Experimental Particle Physics \\ 
& School of Physics and Astronomy, University of Edinburgh, UK. \\ 
& Supervisor Prof. F. Muheim \\ \\
\bf{2008} & Master of Physics with honours \\ 
& School of Physics and Astronomy, University of Edinburgh, UK. \\ 
\end{tabular}
\end{flushleft}
\subsection*{Current position}
\begin{flushleft}
\begin{tabular}{rl}
\bf{2014 - \phantom{2014}} &  Collaborateur Scientifique, Laboratoire de physique des hautes \'energies\\ 
& \'Ecole polytechnique f\'ed\'erale de Lausanne, Lausanne, Switzerland
\end{tabular}
\end{flushleft}

\subsection*{Fellowships and awards}
\begin{flushleft}
\begin{tabular}{rp{13.5cm}}
\bf{2012 - 2014} & Research Fellowship, LHCb experiment \\ 
& CERN, Switzerland \\
\\
\bf{2016}\phantom{ - 2014} & LHCb Early Career Scientist Award  \\ 
& LHCb Collaboration, CERN \\ 
& ``...for having implemented and commissioned the revolutionary changes to the LHC Run-2  high-level trigger, including the first widespread deployment of real-time analysis  techniques in high-energy physics.''
\end{tabular}
\end{flushleft}
\subsection*{Supervision of graduate students}
\begin{flushleft}
\begin{tabular}{rp{13.5cm}}
\bf{2014 - 2018} &  \'Ecole polytechnique f\'ed\'erale de Lausanne, Lausanne, Switzerland \\
& {\bf{PhD student supervisor}} for Vincenzo Battista, EPFL Thesis no. 8848 `Measurement of time-dependent \CP violation in \HepProcess{\PB\to\PDmp\Ppipm} decays and optimisation of flavour tagging algorithms at LHCb'\\
& {\bf{Masters thesis supervisor}} for Marc Huwiler `A search for the decay \HepProcess{\PBzero\to\PDsplus\PDsminus} using multivariate techniques at LHCb'
\end{tabular}
\end{flushleft}
\subsection*{Teaching activities}
\begin{flushleft}
\begin{tabular}{rp{13.5cm}}
\bf{2014 - 2018} &  \'Ecole polytechnique f\'ed\'erale de Lausanne, Lausanne, Switzerland \\
& Course organiser `Introduction to High Energy Physics Software' for Masters and final year undergraduate students  \\ 
& Project Supervisor for final-year undergraduate and Masters student projects.\\
\bf{2013 - 2018} & CERN, Meyrin, Switzerland \\
& Summer student programme supervisor for four students, two of whom have {\textbf{received awards for their work}}. Supervisor for one Masters internship. \\
\bf{2008 - 2012} & University of Edinburgh, UK \\ 
& Tutor, laboratory and workshop demonstrator for introductory physics courses. \\%
  & Laboratory demonstrator for final year undergraduate course `Electronic Methods in the Physical Laboratory'. 
\end{tabular}
\end{flushleft}
\subsection*{Organisation of Scientific Meetings}
\begin{flushleft}
\begin{tabular}{rl}
\bf{2017} & QCD + Heavy Flavour session convenor, Rencontres de Blois  \\
\bf{2015} & Discussion leader, CERN-Fermilab Hadron Collider Physics summer school \\
\bf{2011} & Organising Committee, Young Experimentalists and Theorists Institute, IPPP Durham  \\
\end{tabular}
\end{flushleft}

\subsection*{Leadership Responsibilities} 
\begin{flushleft}
\begin{tabular}{rp{14cm}}
  \bf{2017 - \phantom{2018}} & \textsl{\textbf{Project Leader}, LHCb Trigger} \\
  & As Project Leader, I lead an international team of 9 collaborating institutes responsible for the operation of the present LHCb trigger, and R\&D for future trigger upgrades. \\
  \bf{2015 - 2017} & \textsl{\textbf{Convenor}, LHCb Beauty to Open Charm (B2OC) Working Group}  \\
    & The LHCb physics programme is organised into nine top-level physics working groups. As convenor of the largest working group I was responsible for the physics output of over 60 analysts working on \sim30 analyses. I led the analysers to the publication of 20 peer-reviewed papers including several world-first and most precise measurements. \\
  \bf{2016 - 2017} & \textsl{Deputy Project Leader, LHCb Higher Level Trigger} \\
  \bf{2013 - 2015} & \textsl{Convenor, LHCb B2OC (Time-Dependent) subgroup}\\
  \bf{2012 - 2014} & \textsl{Simulation co-ordinator, LHCb Higher Level Trigger} \\
\end{tabular}
\end{flushleft}

~

\subsection*{Research statement}

My research career (2006 - present) has focused on precision measurements of fundamental particles and matter/antimatter asymmetries with the LHCb experiment. While I have developed a broad range of research outputs in several areas of the LHCb research programme, I have specialised in decay-time-dependent analyses through which I have made world-leading measurements of \CP violation in \PBz and \PBs mesons. In measurements of this kind, the SM can be quantified, for example through measurements of the CKM angle $\gamma$, and NP can be searched for, the determination of $\phi_s$ in \HepProcess{\PBs\to\PJpsi\Pphi} and \HepProcess{\PBs\to\PDsplus\PDsminus} decays being a perfect example. Driven by a desire to improve sensitivities for measurements of $\gamma$ and $\phi_s$, I have played a leading role in both the operation of the present LHCb Run 2 trigger and in defining the trigger for the first LHCb upgrade. My work in these areas have been recognised by the collaboration with a working group convenorship, an early career scientist award and Project Leadership of the LHCb higher level trigger. 

Funding of this project will allow me to take the next step in my career, and to become an established independent scientist with a permanent academic position.
It is my intention to become an internationally recognised leader in precision measurements at collider experiments, and an expert in detector trigger and data acquisition techniques. I am enthusiastic about building and leading a team that will develop my ideas, taking the RTA paradigm to future experiments, and with them derive new insights into the nature of our universe. 
 

\clearpage

\section*{Appendix: All ongoing and submitted grants and funding of the PI}

%\begin{table}[ht]
    \centering
\begin{tabular}{p{3.5cm}|p{3cm}|p{1.4cm}|p{1.5cm}|p{1cm}|p{2.3cm}}
    Project title & Funding Source& Amount (EUR) & Period & Role of PI & Relation to current ERC Proposal \\ \hline\hline
    Precision tests of the Standard Model with doubly-charmed beauty decays & Science \& Technologies Facilities Council (STFC) UK & 751,000 & 10/2019-10/2024 & PI & see text \\ \hline
   Connecting the universe: Bringing Real Time Analysis to Particle Physics and Astronomy  & United Kingdom Research and Innovation (UKRI) & 1,635,664 & 04/2019-04/2023 & PI & see text \\ \hline
    Precision tests of the Standard Model with doubly-charmed beauty decays & Royal Society UK & 951,364 & 10/2019-10/2024 & PI & see text\\ \hline
\end{tabular}
\caption{Submitted grants involving the PI.}
    \label{tab:grants}
\end{table}

\noindent
At present I am not involved in any ongoing grants.  
Table~\ref{tab:grants} lists grants to which I have applied and for which a decision has yet to be made.

Of the three proposals listed, those for the Royal Society University Research fellowship (URF) and the STFC Ernest Rutherford fellowship (ERF) cover similar topics to those addressed in this proposal, however they are considerably less ambitious. The ERF and URF cannot be accepted simultaneously if both are successful. If both this proposal and one of the ERF/URF proposals are successful it would be possible to hold both simultaneously, with the other grant effectively bringing additional funding into the project, enabling an expansion of the scope and support of an additional PDRA to work on RTA commissioning and trigger R\&D for future upgrades. 
The UKRI Future Leaders Fellowship (FLF) has only minor overlap with this proposal, but is complementary as it is designed to develop the RTA paradigm on future large-scale experimental infrastructure (The Square Kilometer Array). If successful this would enable software engineering expertise to enhance and expand the RTA developments in WP1.

\clearpage

\section*{Section C: Early achievements track record}

%\subsection*{Publications} 
I am a named author of more than 400 papers published by the LHCb collaboration, in addition to several few-author papers. The full LHCb collaboration author list includes over 500 individuals (even more in recent papers), and therefore cannot be given here.  
Following the standard procedure in high energy physics, all authors are listed alphabetically.
As my Ph.D.\ supervisor is also a member of the LHCb collaboration, he is a co-author of mine on those publications, but not on my other papers.
A complete list of all my papers, including as-yet unpublished preprints, can be found at 
\begin{center}
    \href{http://inspirehep.net/author/profile/C.Fitzpatrick.1}{http://inspirehep.net/author/profile/C.Fitzpatrick.1}
\end{center}
The most relevant papers for this proposal are:
 \begin{enumerate}
     \setlength\itemsep{1ex}

          \item {\bf{ Measurement of \CP violation in \HepProcess{\PBz\to\PDmp\Ppipm} decays }} \\%
                {R.~Aaij {\it et al.} [ LHCb Collaboration ].}      \\%
                JHEP 06 (2018) 084 \\
                       I initiated this analysis, and together with a PhD student under my supervision, was responsible for the \HepProcess{\PBz} invariant mass fit, the development of the decay time fitting framework modifications necessary to extract the \CP observables, treatment of the opposite-side tagging parameterisation and the decay time resolution determination. I showed that the fit was sensitive to the flavor tagging calibration and did not require that this was measured independently, resulting in reduced systematic uncertainties.

        \item {\bf{Measurement of the CKM angle $\gamma$ from a combination of LHCb results}} \\%
                                   {R.~Aaij {\it et al.} [ LHCb Collaboration ].}      \\%
                  JHEP 12 (2016) 087 \\
            As Convenor of the \LHCb working group that measures $\gamma$, I was responsible for the underlying analyses culminating in this result, which, for the first time, was the most precise single-experiment determination of $\gamma$ from \LHCb.
        
    \item {\bf{Measurement of the \CP-violating phase \phis in \BsToDsDs decays}} \\%
            {R.~Aaij {\it et al.} [ LHCb Collaboration ].} \\%
            Phys. Rev. Lett. 113, 211801 (2014)\\
            I was the lead proponent of this publication, performing the entirety of the analysis from the offline selection onwards.
            I was responsible for taking the analysis and publication through all stages of review, both internal within the LHCb collaboration and after submission to the journal.

    \item {\bf{Prompt charm production in pp collisions at $\sqrt{s}= 7~\TeV$}}         \\%
    {R.~Aaij {\it et al.} [ LHCb Collaboration ].}      \\%
    Nucl. Phys. B871 (2013) 1-20.\\
     I made the \HepProcess{\PDspm} cross-section measurement detailed in this publication in addition to the \PDpm cross-check. This work built upon the \PDspm/\PDpm cross-section ratio I performed with the first \LHCb data as an early measurement, presented at the ICHEP conference in 2010.

    \item {\bf{Measurement of the CP-violating phase $\phi_s$ in the decay \BsToJpsiPhi}} \\%
         {R.~Aaij {\it et al.} [ LHCb Collaboration ].} \\%
         Phys. Rev. Lett. 108, 101803 (2012). \\
         This measurement was the main component of my PhD thesis. I developed the fitting framework, verified the helicity and transversity formalisms, and implemented the Feldman-Cousins statistical procedure used to determine the published result.

  \end{enumerate}

\subsection*{International conference talks} 
I have given the following talks at major international conferences:
  \begin{enumerate}
    \setlength\itemsep{1ex}
          \item {\bf Flavour at LHCb: Recent results and prospects }\\
        Seventh Workshop on Theory, Phenomenology and Experiments in Flavour Physics, Capri, (8 - 10 June 2018)
   \item {\bf Too much of a good thing: How to trigger in a signal-rich environment} \\
        EP/IT Data Science seminar series, CERN, (13 Dec. 2017)
    \item {\bf CP Violation} (review)\\
            $29^{\textrm{th}}$ Rencontres de Blois, Blois, France (28 May - 3 June 2017)
    \item {\bf Combining $\gamma$ at LHCb}\\
            $9^{\textrm{th}}$ International Workshop on the CKM Unitarity Triangle, Mumbai, India (28 Nov - 2 Dec 2016)
    \item {\bf LHCb 2015 Highlights and Status}\\
            $178^{\textrm{th}}$ session of the CERN Council, CERN, Geneva (14-18 Dec 2015)
    \item {\bf Measurement of \CP observables using \BsToDsDs at LHCb}\\
            $8^{\textrm{th}}$ International Workshop on the CKM Unitarity Triangle, Vienna, Austria (8-12 Sept 2014)
    \item {\bf The upgrade of the LHCb trigger system}\\
            Workshop on Intelligent Trackers, Philadelphia, PA (14-16 May 2014)
    \item {\bf Hadronic b decays and the Unitarity triangle angle $\gamma$ at \LHCb}\\
      Rencontres de Moriond, QCD and High Energy Interactions, La Thuile, Italy (22-29 March, 2014)
  \end{enumerate}

I have also taken part in a large number of local and international workshops on flavour physics that are not listed above. 
I was invited to serve as the QCD + Heavy Flavour Session Convenor, Rencontres de Blois 2017, and as discussion leader at the CERN-Fermilab Hadron Collider Physics Summer School, 2015.

\subsubsection*{LHCb Upgrade Trigger R\&D}
The \LHCb Run I and Run II trigger consists of a level-0 hardware on-detector stage that must reduce the rate of \pp crossings to the $1~\MHz$ limit at which the present detector readout operates, at which point a second software Higher Level Trigger (HLT) stage reduces this rate further to satisfy offline data processing and storage requirements. From Run III onwards LHCb will operate at five times the Run I instantaneous luminosity with several subdetector upgrades. As a CERN fellow I performed studies of what this increased luminosity would mean for the trigger~\cite{LHCb-PUB-2014-027}. My studies showed that the most cost-effective solution would be to read out the full detector at the collider bunch crossing frequency of $30\MHz$ directly into a software trigger where the full event can be reconstructed. As a result of this work, \LHCb chose this triggerless readout with a fully software trigger as the upgrade design~\cite{CERN-LHCC-2014-016}. \textbf{For this work I was made deputy project leader of the HLT responsible for the upgrade in 2016.}

\subsubsection*{The LHCb Run II Trigger commissioning and operations}
The LHCb Run II HLT serves both as a flexible software trigger for the LHCb physics programme, and as a testbed for the upcoming upgrade which will take data in 2021.
However, in Run II the signals needed by the broad LHCb physics programme are still subject to the $1~\MHz$ readout limit.
As a chercheur scientifique at EPFL and deputy project leader of the trigger, I understood the importance of maximising the efficient use of this rate, and developed a method to optimise the level-0 trigger configuration using a genetic algorithm. This optimisation maximises the signal efficiency subject to the readout constraint for the entire LHCb physics programme, taking into account the physics priorities of the experiment and detector deadtime. \textbf{My work as part of the HLT team on commissioning the Run II trigger was recognised with an \LHCb early career scientist award}, and as of 2017 I have been made Project Leader for the HLT, where I direct both the operational activities of the Run II HLT, and the construction of the upgrade trigger.

\end{document}  
