%Boilerplate LHC is great. the largest discovery machine ever built by humankind.
Experiments at the CERN laboratory in Geneva, studying collisions from the Large Hadron Collider (LHC)~\cite{LHC2008}, have verified the predictive power of the Standard Model (SM) of particle physics. % Characters: LHC or ATLAS Experiment or you and your collaborators? How has this been shown? Establish that this is a thoroughly proven experiment that has already done nobel-prize level work.
% DM is an important problem
Nevertheless, the SM must be an incomplete theory.
It describes all known fundamental particles and their non-gravitational interactions,
but it completely lacks any particle consistent with the astrophysical evidence for Dark Matter (DM)~\cite{Bertone:2016nfn}.
The \textbf{failure of the SM to account for DM}, which in the universe is about five times as abundant as the matter described by the SM, remains one of the most important puzzles in high energy physics and astrophysics research.

% known (not ordinary); by the SM
%  evidenced from astrophysical observations indicates that the SM is an incomplete theory.
% no shortage of ideas, but the puzzle must be resolved with data
% Astrophysical observations indicate that DM is about five times as abundant as ordinary matter in the universe.

Solving this puzzle requires new experimental data. 
Absent clues from the LHC and other particle physics experiments, 
%[AB: rephrase the following] we mainly know what we know about DM characteristics from astrophysics, interacting only gravitationally or with some very feeble interactions with SM matter.
astrophysics observations provide evidence of DM interacting gravitationally, and hints of very feeble interactions between DM and SM matter~\cite{Bernal:2017kxu,Steigman:2012nb}. 
\textbf{Probing for these interactions requires larger and larger datasets to reveal.} 
%[CD: "probing for"? If this is an ABddition I trust it but if not maybe it's "Probing these interactions]
%[CD: The sentences below are attempts at conveying complementarity early]
LHC experiments can play a key role in discovering DM-SM interactions, complementary to other particle physics and astrophysics experiments~\cite{Boveia:2018yeb}, since DM particles can emerge from high-rate collisions of SM particles. 
At the same time, LHC experiments and many modern particle physics experiments face a data acquisition challenge.
%Original sentence:
%At the same time, the enormous data rates of modern particle physics experiments present a data acquisition challenge.
%With traditional methods, it is not possible to record and store these large datasets when the rare processes are buried in high-rate backgrounds. 
%%Maybe add: the LHC is sensitive to DM if we assume that there are some interactions…
%Other attempts:
%With the largest-ever dataset of proton-proton collisions that could have produced DM particles, 
%At the same time, the enormous data rates of the LHC and of many modern particle physics experiments present a data acquisition challenge.
% if DM interacts with SM particles, these interactions must be very feeble and/or the experimental signals of DM are subtle, requiring large datasets to reveal.
With traditional data acquisition methods, it is not possible to record and store the extremely large datasets needed to discover DM or other rare processes buried in substantially larger backgrounds. 
%Maybe add: the LHC is sensitive to DM if we assume that there are some interactions…

As a senior lecturer at Lund University, I will lead a research team to search for signs of DM and other new phenomena in LHC data to be collected in 2021--2026 by the ATLAS experiment. 
As described in Part B1, our work is divided in five interconnected Work Packages (WPs). Within this proposal:  
%Earlier text
%We will deploy a \textbf{new data-taking paradigm} that significantly increases the discovery potential of LHC data for the entire ATLAS experiment, 
%"even though" / "when" is crap 
%even though the LHC center-of-mass energy and dataset size are expected to just be comparable to previous data-taking runs. 
%We will use data collected with new techniques to \textbf{search for signals of broad classes of compelling DM models} and new phenomena that are rare or have so far been neglected, leading to discoveries or constraints with an impact on the global DM community.  
%We will disseminate the results of this research and its innovations through working groups and cross-experimental collaborations of theorists and experimentalists from collider and astrophysics experiments, and to others outside academia. 
%I already said particle physics and astrophysics above
%particle physics and astrophysics. 

\begin{description}

%In \textsc{Realdark} we will use novel data taking techniques (developed in Work Package 1) to \textbf{further the systematic exploration} of the hadronic decays of the DM mediator as in the left-hand side of Fig.~\ref{fig:feynman} (WP2-3). 

\item[(WP1)] We will \textbf{extend the capabilities of the ATLAS trigger system} with a comprehensive set of real-time analysis techniques that significantly increases the discovery potential of LHC data for the entire ATLAS experiment. This builds on preliminary work done within my StG to deploy the proof-of-principle for a real-time analysis technique in ATLAS that allows to record orders of magnitude more data, called Trigger-object Level Analysis (TLA)~\cite{Aaboud:2018fzt}.
\item[(WP2)] We will \textbf{commission the upgraded ATLAS trigger with early Run-3 data}, using methods and searches that already have a solid basis from Run-2. We will rely on my extensive experience in reconstruction and calibration~\cite{Aad:2014bia,Aaboud:2018kfi}, and past involvement in early measurements and searches in every LHC start-up phase~\cite{Doglioni:2011ema,Aad:2014aqa,ATLAS:2015nsi}. %that I brought to Lund? 
\item[(WP3-WP4)] we will using the LHC dataset recorded with novel data taking techniques to perform \textbf{new searches for broad classes of DM models}. The first set of searches targets WIMP DM mediators and probes electroweak-scale mediator masses not fully explored by traditional searches (WP3), while the second set of searches targets dark QCD models, recovering sensitivity to signals that escape traditional detector reconstruction techniques. The choice of models matches the expertise in the experimental and theoretical particle physics divisions in Lund, and my background as one of the co-organizers of the LHC Dark Matter Working Group. 
\item[(WP5)] \textbf{disseminate and communicate physics results and tools}, with a broad impact for the broader DM and experimental community. This will be supported by my coordination role in the \href{https://hepsoftwarefoundation.org}{HEP Software Foundation} and by my involvement in the \href{https://projectescape.eu}{ESCAPE project}.

\end{description}

This five-year program will advance the state-of-the-art in data-taking and data-processing of enormous datasets at scientific experiments and shed further light on one of the greatest mysteries of our universe. 





