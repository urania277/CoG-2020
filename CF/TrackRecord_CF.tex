\subsection*{Publications} 
I am a named author of more than 400 papers published by the LHCb collaboration, in addition to several few-author papers. The full LHCb collaboration author list includes over 500 individuals (even more in recent papers), and therefore cannot be given here.  
Following the standard procedure in high energy physics, all authors are listed alphabetically.
As my Ph.D.\ supervisor is also a member of the LHCb collaboration, he is a co-author of mine on those publications, but not on my other papers.
A complete list of all my papers, including as-yet unpublished preprints, can be found at 
\begin{center}
    \href{http://inspirehep.net/author/profile/C.Fitzpatrick.1}{http://inspirehep.net/author/profile/C.Fitzpatrick.1}
\end{center}
The most relevant papers for this proposal are:
 \begin{enumerate}
     \setlength\itemsep{1ex}

          \item {\bf{ Measurement of \CP violation in \HepProcess{\PBz\to\PDmp\Ppipm} decays }} \\%
                {R.~Aaij {\it et al.} [ LHCb Collaboration ].}      \\%
                JHEP 06 (2018) 084 \\
                       I initiated this analysis, and together with a PhD student under my supervision, was responsible for the \HepProcess{\PBz} invariant mass fit, the development of the decay time fitting framework modifications necessary to extract the \CP observables, treatment of the opposite-side tagging parameterisation and the decay time resolution determination. I showed that the fit was sensitive to the flavor tagging calibration and did not require that this was measured independently, resulting in reduced systematic uncertainties.

        \item {\bf{Measurement of the CKM angle $\gamma$ from a combination of LHCb results}} \\%
                                   {R.~Aaij {\it et al.} [ LHCb Collaboration ].}      \\%
                  JHEP 12 (2016) 087 \\
            As Convenor of the \LHCb working group that measures $\gamma$, I was responsible for the underlying analyses culminating in this result, which, for the first time, was the most precise single-experiment determination of $\gamma$ from \LHCb.
        
    \item {\bf{Measurement of the \CP-violating phase \phis in \BsToDsDs decays}} \\%
            {R.~Aaij {\it et al.} [ LHCb Collaboration ].} \\%
            Phys. Rev. Lett. 113, 211801 (2014)\\
            I was the lead proponent of this publication, performing the entirety of the analysis from the offline selection onwards.
            I was responsible for taking the analysis and publication through all stages of review, both internal within the LHCb collaboration and after submission to the journal.

    \item {\bf{Prompt charm production in pp collisions at $\sqrt{s}= 7~\TeV$}}         \\%
    {R.~Aaij {\it et al.} [ LHCb Collaboration ].}      \\%
    Nucl. Phys. B871 (2013) 1-20.\\
     I made the \HepProcess{\PDspm} cross-section measurement detailed in this publication in addition to the \PDpm cross-check. This work built upon the \PDspm/\PDpm cross-section ratio I performed with the first \LHCb data as an early measurement, presented at the ICHEP conference in 2010.

    \item {\bf{Measurement of the CP-violating phase $\phi_s$ in the decay \BsToJpsiPhi}} \\%
         {R.~Aaij {\it et al.} [ LHCb Collaboration ].} \\%
         Phys. Rev. Lett. 108, 101803 (2012). \\
         This measurement was the main component of my PhD thesis. I developed the fitting framework, verified the helicity and transversity formalisms, and implemented the Feldman-Cousins statistical procedure used to determine the published result.

  \end{enumerate}

\subsection*{International conference talks} 
I have given the following talks at major international conferences:
  \begin{enumerate}
    \setlength\itemsep{1ex}
          \item {\bf Flavour at LHCb: Recent results and prospects }\\
        Seventh Workshop on Theory, Phenomenology and Experiments in Flavour Physics, Capri, (8 - 10 June 2018)
   \item {\bf Too much of a good thing: How to trigger in a signal-rich environment} \\
        EP/IT Data Science seminar series, CERN, (13 Dec. 2017)
    \item {\bf CP Violation} (review)\\
            $29^{\textrm{th}}$ Rencontres de Blois, Blois, France (28 May - 3 June 2017)
    \item {\bf Combining $\gamma$ at LHCb}\\
            $9^{\textrm{th}}$ International Workshop on the CKM Unitarity Triangle, Mumbai, India (28 Nov - 2 Dec 2016)
    \item {\bf LHCb 2015 Highlights and Status}\\
            $178^{\textrm{th}}$ session of the CERN Council, CERN, Geneva (14-18 Dec 2015)
    \item {\bf Measurement of \CP observables using \BsToDsDs at LHCb}\\
            $8^{\textrm{th}}$ International Workshop on the CKM Unitarity Triangle, Vienna, Austria (8-12 Sept 2014)
    \item {\bf The upgrade of the LHCb trigger system}\\
            Workshop on Intelligent Trackers, Philadelphia, PA (14-16 May 2014)
    \item {\bf Hadronic b decays and the Unitarity triangle angle $\gamma$ at \LHCb}\\
      Rencontres de Moriond, QCD and High Energy Interactions, La Thuile, Italy (22-29 March, 2014)
  \end{enumerate}

I have also taken part in a large number of local and international workshops on flavour physics that are not listed above. 
I was invited to serve as the QCD + Heavy Flavour Session Convenor, Rencontres de Blois 2017, and as discussion leader at the CERN-Fermilab Hadron Collider Physics Summer School, 2015.

\subsubsection*{LHCb Upgrade Trigger R\&D}
The \LHCb Run I and Run II trigger consists of a level-0 hardware on-detector stage that must reduce the rate of \pp crossings to the $1~\MHz$ limit at which the present detector readout operates, at which point a second software Higher Level Trigger (HLT) stage reduces this rate further to satisfy offline data processing and storage requirements. From Run III onwards LHCb will operate at five times the Run I instantaneous luminosity with several subdetector upgrades. As a CERN fellow I performed studies of what this increased luminosity would mean for the trigger~\cite{LHCb-PUB-2014-027}. My studies showed that the most cost-effective solution would be to read out the full detector at the collider bunch crossing frequency of $30\MHz$ directly into a software trigger where the full event can be reconstructed. As a result of this work, \LHCb chose this triggerless readout with a fully software trigger as the upgrade design~\cite{CERN-LHCC-2014-016}. \textbf{For this work I was made deputy project leader of the HLT responsible for the upgrade in 2016.}

\subsubsection*{The LHCb Run II Trigger commissioning and operations}
The LHCb Run II HLT serves both as a flexible software trigger for the LHCb physics programme, and as a testbed for the upcoming upgrade which will take data in 2021.
However, in Run II the signals needed by the broad LHCb physics programme are still subject to the $1~\MHz$ readout limit.
As a chercheur scientifique at EPFL and deputy project leader of the trigger, I understood the importance of maximising the efficient use of this rate, and developed a method to optimise the level-0 trigger configuration using a genetic algorithm. This optimisation maximises the signal efficiency subject to the readout constraint for the entire LHCb physics programme, taking into account the physics priorities of the experiment and detector deadtime. \textbf{My work as part of the HLT team on commissioning the Run II trigger was recognised with an \LHCb early career scientist award}, and as of 2017 I have been made Project Leader for the HLT, where I direct both the operational activities of the Run II HLT, and the construction of the upgrade trigger.