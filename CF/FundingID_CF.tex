\begin{table}[ht]
    \centering
\begin{tabular}{p{3.5cm}|p{3cm}|p{1.4cm}|p{1.5cm}|p{1cm}|p{2.3cm}}
    Project title & Funding Source& Amount (EUR) & Period & Role of PI & Relation to current ERC Proposal \\ \hline\hline
    Precision tests of the Standard Model with doubly-charmed beauty decays & Science \& Technologies Facilities Council (STFC) UK & 751,000 & 10/2019-10/2024 & PI & see text \\ \hline
   Connecting the universe: Bringing Real Time Analysis to Particle Physics and Astronomy  & United Kingdom Research and Innovation (UKRI) & 1,635,664 & 04/2019-04/2023 & PI & see text \\ \hline
    Precision tests of the Standard Model with doubly-charmed beauty decays & Royal Society UK & 951,364 & 10/2019-10/2024 & PI & see text\\ \hline
\end{tabular}
\caption{Submitted grants involving the PI.}
    \label{tab:grants}
\end{table}

\noindent
At present I am not involved in any ongoing grants.  
Table~\ref{tab:grants} lists grants to which I have applied and for which a decision has yet to be made.

Of the three proposals listed, those for the Royal Society University Research fellowship (URF) and the STFC Ernest Rutherford fellowship (ERF) cover similar topics to those addressed in this proposal, however they are considerably less ambitious. The ERF and URF cannot be accepted simultaneously if both are successful. If both this proposal and one of the ERF/URF proposals are successful it would be possible to hold both simultaneously, with the other grant effectively bringing additional funding into the project, enabling an expansion of the scope and support of an additional PDRA to work on RTA commissioning and trigger R\&D for future upgrades. 
The UKRI Future Leaders Fellowship (FLF) has only minor overlap with this proposal, but is complementary as it is designed to develop the RTA paradigm on future large-scale experimental infrastructure (The Square Kilometer Array). If successful this would enable software engineering expertise to enhance and expand the RTA developments in WP1.