
\section*{Section B: Curriculum Vitae} 
\subsection*{Personal information}
\begin{tabular}{rlrl}
{\bf Name} & Dr.~Conor Fitzpatrick & {\bf Date of birth} & 03/10/1982 \\
{\bf Nationality} & Irish \\
{\bf ORCiD} & \hyperlink{https://orcid.org/0000-0003-3674-0812}{0000-0003-3674-0812} & {\bf inSPIRE} & \hyperlink{https://inspirehep.net/author/profile/C.Fitzpatrick.1}{C.Fitzpatrick.1}

\end{tabular}
\subsection*{Education}
\begin{flushleft}
\begin{tabular}{rl}
\bf{2012} & PhD, Experimental Particle Physics \\ 
& School of Physics and Astronomy, University of Edinburgh, UK. \\ 
& Supervisor Prof. F. Muheim \\ \\
\bf{2008} & Master of Physics with honours \\ 
& School of Physics and Astronomy, University of Edinburgh, UK. \\ 
\end{tabular}
\end{flushleft}
\subsection*{Current position}
\begin{flushleft}
\begin{tabular}{rl}
\bf{2014 - \phantom{2014}} &  Collaborateur Scientifique, Laboratoire de physique des hautes \'energies\\ 
& \'Ecole polytechnique f\'ed\'erale de Lausanne, Lausanne, Switzerland
\end{tabular}
\end{flushleft}

\subsection*{Fellowships and awards}
\begin{flushleft}
\begin{tabular}{rp{13.5cm}}
\bf{2012 - 2014} & Research Fellowship, LHCb experiment \\ 
& CERN, Switzerland \\
\\
\bf{2016}\phantom{ - 2014} & LHCb Early Career Scientist Award  \\ 
& LHCb Collaboration, CERN \\ 
& ``...for having implemented and commissioned the revolutionary changes to the LHC Run-2  high-level trigger, including the first widespread deployment of real-time analysis  techniques in high-energy physics.''
\end{tabular}
\end{flushleft}
\subsection*{Supervision of graduate students}
\begin{flushleft}
\begin{tabular}{rp{13.5cm}}
\bf{2014 - 2018} &  \'Ecole polytechnique f\'ed\'erale de Lausanne, Lausanne, Switzerland \\
& {\bf{PhD student supervisor}} for Vincenzo Battista, EPFL Thesis no. 8848 `Measurement of time-dependent \CP violation in \HepProcess{\PB\to\PDmp\Ppipm} decays and optimisation of flavour tagging algorithms at LHCb'\\
& {\bf{Masters thesis supervisor}} for Marc Huwiler `A search for the decay \HepProcess{\PBzero\to\PDsplus\PDsminus} using multivariate techniques at LHCb'
\end{tabular}
\end{flushleft}
\subsection*{Teaching activities}
\begin{flushleft}
\begin{tabular}{rp{13.5cm}}
\bf{2014 - 2018} &  \'Ecole polytechnique f\'ed\'erale de Lausanne, Lausanne, Switzerland \\
& Course organiser `Introduction to High Energy Physics Software' for Masters and final year undergraduate students  \\ 
& Project Supervisor for final-year undergraduate and Masters student projects.\\
\bf{2013 - 2018} & CERN, Meyrin, Switzerland \\
& Summer student programme supervisor for four students, two of whom have {\textbf{received awards for their work}}. Supervisor for one Masters internship. \\
\bf{2008 - 2012} & University of Edinburgh, UK \\ 
& Tutor, laboratory and workshop demonstrator for introductory physics courses. \\%
  & Laboratory demonstrator for final year undergraduate course `Electronic Methods in the Physical Laboratory'. 
\end{tabular}
\end{flushleft}
\subsection*{Organisation of Scientific Meetings}
\begin{flushleft}
\begin{tabular}{rl}
\bf{2017} & QCD + Heavy Flavour session convenor, Rencontres de Blois  \\
\bf{2015} & Discussion leader, CERN-Fermilab Hadron Collider Physics summer school \\
\bf{2011} & Organising Committee, Young Experimentalists and Theorists Institute, IPPP Durham  \\
\end{tabular}
\end{flushleft}

\subsection*{Leadership Responsibilities} 
\begin{flushleft}
\begin{tabular}{rp{14cm}}
  \bf{2017 - \phantom{2018}} & \textsl{\textbf{Project Leader}, LHCb Trigger} \\
  & As Project Leader, I lead an international team of 9 collaborating institutes responsible for the operation of the present LHCb trigger, and R\&D for future trigger upgrades. \\
  \bf{2015 - 2017} & \textsl{\textbf{Convenor}, LHCb Beauty to Open Charm (B2OC) Working Group}  \\
    & The LHCb physics programme is organised into nine top-level physics working groups. As convenor of the largest working group I was responsible for the physics output of over 60 analysts working on \sim30 analyses. I led the analysers to the publication of 20 peer-reviewed papers including several world-first and most precise measurements. \\
  \bf{2016 - 2017} & \textsl{Deputy Project Leader, LHCb Higher Level Trigger} \\
  \bf{2013 - 2015} & \textsl{Convenor, LHCb B2OC (Time-Dependent) subgroup}\\
  \bf{2012 - 2014} & \textsl{Simulation co-ordinator, LHCb Higher Level Trigger} \\
\end{tabular}
\end{flushleft}

~

\subsection*{Research statement}

My research career (2006 - present) has focused on precision measurements of fundamental particles and matter/antimatter asymmetries with the LHCb experiment. While I have developed a broad range of research outputs in several areas of the LHCb research programme, I have specialised in decay-time-dependent analyses through which I have made world-leading measurements of \CP violation in \PBz and \PBs mesons. In measurements of this kind, the SM can be quantified, for example through measurements of the CKM angle $\gamma$, and NP can be searched for, the determination of $\phi_s$ in \HepProcess{\PBs\to\PJpsi\Pphi} and \HepProcess{\PBs\to\PDsplus\PDsminus} decays being a perfect example. Driven by a desire to improve sensitivities for measurements of $\gamma$ and $\phi_s$, I have played a leading role in both the operation of the present LHCb Run 2 trigger and in defining the trigger for the first LHCb upgrade. My work in these areas have been recognised by the collaboration with a working group convenorship, an early career scientist award and Project Leadership of the LHCb higher level trigger. 

Funding of this project will allow me to take the next step in my career, and to become an established independent scientist with a permanent academic position.
It is my intention to become an internationally recognised leader in precision measurements at collider experiments, and an expert in detector trigger and data acquisition techniques. I am enthusiastic about building and leading a team that will develop my ideas, taking the RTA paradigm to future experiments, and with them derive new insights into the nature of our universe. 
 